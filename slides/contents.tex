%%%%%%%%%%%%%%%%%%%%%%%%%%%%%%%%%%%%%%%%%%%%%%%%%%
\begin{frame}{Software Craftsmanship}

\Large{Was bedeutet das?}


\begin{itemize}
\onslide+<2->
	\item Clean Code
	
\onslide+<3->
	\item Vorsicht: Lokale Optimierung!
	
\onslide+<4->
	\item Blick auf das große Ganze
	
\onslide+<5->
	\item Gesamter Entwicklungsprozess: Anforderung bis Deployment
	
\onslide+<6->
	\item Kundenbedürfnisse sind zentral
	
\end{itemize}


\end{frame}


%%%%%%%%%%%%%%%%%%%%%%%%%%%%%%%%%%%%%%%%%%%%%%%%%%
\begin{frame}{Software Craftsmanship}

\begin{center}
\Large
\glqq{}Das Richtige richtig tun.\grqq{}
\end{center}

\end{frame}

%%%%%%%%%%%%%%%%%%%%%%%%%%%%%%%%%%%%%%%%%%%%%%%%%%
\begin{frame}{Handwerkergilden}

\begin{center}
\includegraphics[width=8cm]{bilder/Softwerkskammer_WappenMitText.png}
\end{center}

\end{frame}

%%%%%%%%%%%%%%%%%%%%%%%%%%%%%%%%%%%%%%%%%%%%%%%%%%
\begin{frame}{Die Walz}

\begin{itemize}
\item Wichtiger Ausbildungsschritt

\item Gibt's auch heute noch
\end{itemize}

\end{frame}

%%%%%%%%%%%%%%%%%%%%%%%%%%%%%%%%%%%%%%%%%%%%%%%%%%
\begin{frame}{Craftsman Swaps}

\onslide+<2->

\begin{itemize}
\item Zwei Entwickler besuchen sich gegenseitig
\begin{itemize}
\item zeitgleich
\item zeitversetzt
\end{itemize}

\onslide+<3->

\item Ein Entwickler besucht einen anderen
\begin{itemize}
\item Ohne Gegenbesuch
\end{itemize}

\onslide+<4->

\item Die Walz
\begin{itemize}
\item Für Serientäter
\end{itemize}

\end{itemize}

\end{frame}

%%%%%%%%%%%%%%%%%%%%%%%%%%%%%%%%%%%%%%%%%%%%%%%%%%
\begin{frame}{In Deutschland?!}

\onslide+<2->

\begin{center}
\Large
Ja, in Deutschland!
\end{center}

\end{frame}

%%%%%%%%%%%%%%%%%%%%%%%%%%%%%%%%%%%%%%%%%%%%%%%%%%
\begin{frame}{Bekannte Craftsmen}

\begin{itemize}
\item Ralf Westphal
\begin{itemize}
\item Walz im Sommer 
\item
\end{itemize}


\item Daniel Temme
\begin{itemize}
\item Walz im Sommer 2013
\item
\end{itemize}

\item Peter Kofler
\begin{itemize}
\item \glqq{}Pair Programming Tour\grqq{} im  
\end{itemize}

\item Matthias Mayer (msgGillardon AG) und Christian Burkhardt (cinovo AG)
\begin{itemize}
\item Austausch geplant für Dez. 2013 / Jan. 2014
\end{itemize}


\end{itemize}

\end{frame}

%%%%%%%%%%%%%%%%%%%%%%%%%%%%%%%%%%%%%%%%%%%%%%%%%%
{
\usebackgroundtemplate{\includegraphics[width=\paperwidth,height=\paperheight]{background-slide.png}}
\begin{frame}{Vielen Dank!}

        Folien auf GitHub:
        \vspace{-0.8em}
        \begin{center}
                \url{https://github.com/NicoleRauch/CraftsmanSwaps}
        \end{center}

        \begin{block}{Nicole Rauch}
        \begin{description}[Twitterxx]
                \item[E-Mail]  \href{mailto:nicole.rauch@msg-gillardon.de}{\texttt{nicole.rauch@msg-gillardon.de}}
                \item[Twitter] \href{http://twitter.com/NicoleRauch}{\texttt{@NicoleRauch}}
        \end{description}
        \end{block}
\end{frame}
}

%%%%%%%%%%%%%%%%%%%%%%%%%%%%%%%%%%%%%%%%%%%%%%%%%%
%\begin{frame}{Quellen}
%
%\begin{itemize}
%\item Kompliziert.jpg: 
%\url{http://chestofbooks.com/crafts/metal/Elementary-Metal-Work/Nails-And-Nailed-Strips.html}
%\item Formeln.jpg: 
%\url{http://depts.washington.edu/ecnboard/wordpress/wp-content/uploads/2011/01/math_image-300x269.jpg}
%\item Namen.jpg: 
%\end{itemize}
%
%\end{frame}
