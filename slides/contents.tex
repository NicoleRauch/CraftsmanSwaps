%%%%%%%%%%%%%%%%%%%%%%%%%%%%%%%%%%%%%%%%%%%%%%%%%%
\begin{frame}{Software Craftsmanship}

\Large{Was bedeutet das?}


\begin{itemize}
\onslide+<2->
	\item Clean Code, Katas, Dojos, $\ldots$
	
\onslide+<3->
	\item Das ist nicht genug!
	
\onslide+<4->
	\item Blick auf das große Ganze
	
\onslide+<5->
	\item Gesamter Entwicklungsprozess: Anforderung bis Deployment
	
\onslide+<6->
	\item Kundenbedürfnisse sind zentral
	
\end{itemize}


\end{frame}


%%%%%%%%%%%%%%%%%%%%%%%%%%%%%%%%%%%%%%%%%%%%%%%%%%
\begin{frame}{Software Craftsmanship}

\begin{center}
\Large
\glqq{}Das Richtige richtig tun.\grqq{}
\end{center}

\end{frame}

%%%%%%%%%%%%%%%%%%%%%%%%%%%%%%%%%%%%%%%%%%%%%%%%%%
\begin{frame}{Software Craftsmanship}

\begin{center}
\Large
\glqq{}Craftsmen are people who care.\grqq{}
\end{center}

\end{frame}

%%%%%%%%%%%%%%%%%%%%%%%%%%%%%%%%%%%%%%%%%%%%%%%%%%
\begin{frame}{Handwerkergilden}

\begin{center}
\includegraphics[width=8cm]{bilder/Softwerkskammer_WappenMitText.png}
\end{center}

\end{frame}

%%%%%%%%%%%%%%%%%%%%%%%%%%%%%%%%%%%%%%%%%%%%%%%%%%
\begin{frame}{Die Walz}

\begin{itemize}
\item Wichtiger Ausbildungsschritt im Handwerk

\item Gibt's auch heute noch
\end{itemize}

\end{frame}

%%%%%%%%%%%%%%%%%%%%%%%%%%%%%%%%%%%%%%%%%%%%%%%%%%
\begin{frame}{Craftsman Swaps}

\onslide+<2->

\begin{itemize}
\item Zwei Entwickler besuchen sich gegenseitig
\begin{itemize}
\item zeitgleich
\item zeitversetzt
\end{itemize}

\onslide+<3->

\item Ein Entwickler besucht einen anderen
\begin{itemize}
\item Ohne Gegenbesuch
\end{itemize}

\onslide+<4->

\item Die \glqq Walz\grqq{} oder auch \glqq Journeyman Tour\grqq{}
\begin{itemize}
\item Für Serientäter
\end{itemize}

\end{itemize}

\end{frame}

%%%%%%%%%%%%%%%%%%%%%%%%%%%%%%%%%%%%%%%%%%%%%%%%%%
\begin{frame}{In Deutschland?!}

\onslide+<2->

\begin{center}
\Large
Ja, in Deutschland!
\end{center}

\end{frame}

%%%%%%%%%%%%%%%%%%%%%%%%%%%%%%%%%%%%%%%%%%%%%%%%%%
\begin{frame}{Bekannte Craftsmen}

\begin{itemize}
\item Ralf Westphal
\begin{itemize}
\item \glqq Walz\grqq{} im Sommer 2013
\item \url{http://www.dotnetpro.de/Grafix/OnlineArticles/westphal_walz.pdf}
\end{itemize}


\item Daniel Temme
\begin{itemize}
\item \glqq Walz\grqq{} im Sommer 2013
\item \url{http://danieltemme.blogspot.de/2013/08/journeyman-weeks.html}
\end{itemize}

\item Peter Kofler
\begin{itemize}
\item \glqq{}Pair Programming Tour\grqq{} seit September 2013
\item \url{http://blog.code-cop.org/search/label/CodeCopTour}
\end{itemize}

\item Matthias Mayer (msgGillardon AG) und Christian Burkhardt (cinovo AG)
\begin{itemize}
\item Austausch im Dez. 2013 / Jan. 2014
\end{itemize}


\end{itemize}

\end{frame}


%%%%%%%%%%%%%%%%%%%%%%%%%%%%%%%%%%%%%%%%%%%%%%%%%%
%%%%%%%%%%%%%%%%%%%%%%%%%%%%%%%%%%%%%%%%%%%%%%%%%%
\begin{frame}{Craftsman Swap - Zur Person}
 
 
\begin{itemize}
\item C++ Entwickler im Bereich Bankensoftware Rechenkerne
\item fachliche und algorithmische Programmierung
\item keine Datenbanken
\item keine Oberflächen
\item Visual Studio
\item Rechenkernanbindung aus Java mittels JNI-Schnittstelle
\item Eclipse und Java im täglichen Gebrauch, aber nur rudimentär
 
\end{itemize}
 
 
\end{frame}
 
%%%%%%%%%%%%%%%%%%%%%%%%%%%%%%%%%%%%%%%%%%%%%%%%%%
\begin{frame}{Craftsman Swap - Durchführung}
 
\begin{itemize}
                \item Cinovo AG, Stuttgart
                \item ca. 16 Mitarbeiter
                \item Softwarelösung für Banken
                \item Java, JavaScript
\end{itemize}
              
\begin{itemize}
                \item msgGillardon AG, Bretten
                \item ca. 400 Mitarbeiter
                \item Produkte und Projekte rund um Finanzsoftware
                \item C++ Rechenkerne, Java-Applikationen
\end{itemize}
 
 
\end{frame}
 
 
%%%%%%%%%%%%%%%%%%%%%%%%%%%%%%%%%%%%%%%%%%%%%%%%%%
\begin{frame}{Craftsman Swap - Durchführung}
 
\begin{itemize}
                \item jeweils 7 Arbeitstage
                \item Mein Austauschpartner war hauptsächlich mein \glqq Begleiter\grqq{}
                \item bei uns: viele Besprechungen
                \item ich war \glqq Besucher\grqq{} an vielen Arbeitsplätzen
                \item bei ihm: selbständiges Arbeiten
\end{itemize}
 
 
\end{frame}
 
 
%%%%%%%%%%%%%%%%%%%%%%%%%%%%%%%%%%%%%%%%%%%%%%%%%%
\begin{frame}{Hilfreiches}
 
\begin{itemize}
 
\item Zum Einstieg Vorstellung der Software, Architektur, Tools, Abteilung
\begin{itemize}
\item Schnelle Einarbeitung möglich
\item besseres Verständnis für Probleme
\end{itemize}
 
\onslide+<2->
\item Unternehmensauswahl ist wichtig
\begin{itemize}
\item fachliche oder technische Ähnlichkeit von Vorteil
\item Bei zu geringer Korrelation zwischen den Unternehmen / Arbeitsweisen könnte sich Frustration einstellen.
\end{itemize}
 
\end{itemize}
 
\end{frame}
 
%%%%%%%%%%%%%%%%%%%%%%%%%%%%%%%%%%%%%%%%%%%%%%%%%%
\begin{frame}{Persönliche Erfahrungen}
 
\begin{itemize}
\onslide+<2->
\item Java, Web-Programmierung, Konfiguration / Deployment
 
\onslide+<3->
\item Eclipse Shortcuts
 
\onslide+<4->
\item Ideen zu Qualitätsverbesserungen
\begin{itemize}
\item Quality-Friday
\item \glqq Code Complete\grqq{}-Themen als Ideengeber
\end{itemize}
 
\onslide+<5->
\item Testen
\begin{itemize}
\item  ist schwierig
\item darf aber nicht vernachlässigt werden
\item größere Refactorings ohne Tests sind risikobehaftet und gehen meistens nicht auf Anhieb gut
\end{itemize}
 
\onslide+<6->
\item Automatisierung
\begin{itemize}
\item Steigerung der Effizienz
\item Vereinfachung Arbeitsabläufe
\item Zusammenspiel zwischen Ticketsystem, Versionsverwaltung, Releasenotesgenerierung
\end{itemize}
 
\onslide+<7->
\item Weniger Besprechung ist machnmal mehr
 
 
\end{itemize}
 
\end{frame}
 
%%%%%%%%%%%%%%%%%%%%%%%%%%%%%%%%%%%%%%%%%%%%%%%%%%
\begin{frame}{Fazit}
 
\begin{itemize}
\onslide+<2->
\item ca. fünf Arbeitstage für ersten Swap ausreichend
\begin{itemize}
\item Einblicke in Themen
\item Verständnis für Software, Arbeitsweise, Technik
\end{itemize}
 
\onslide+<3->
\item eigener Rechner ist von Vorteil
\begin{itemize}
\item eigenständiges Lösen von kleinen Aufgaben
\item Begriffe googeln
\item Ausprobieren und Vertiefen neuer Ideen
\end{itemize}
 
\onslide+<4->
\item Arbeitsplatz Hopping ermöglicht viele Einblicke
\begin{itemize}
\item Einblick in unterschiedliche Arbeitsweisen
\item von jedem Mitarbeiter kann man etwas lernen
\end{itemize}
 
\end{itemize}
 
\end{frame}


%%%%%%%%%%%%%%%%%%%%%%%%%%%%%%%%%%%%%%%%%%%%%%%%%%
\begin{frame}{Rechtliches}

\begin{itemize}
\item Kurz-Besucher: Praktikumsvertrag

\item Craftsman Swap: Dienstleistungsvertrag (Time \& Material)

\item Wichtig: Verträge ohne Bezahlung

\end{itemize}

\end{frame}


%%%%%%%%%%%%%%%%%%%%%%%%%%%%%%%%%%%%%%%%%%%%%%%%%%
\begin{frame}{Und jetzt Ihr!}

\begin{itemize}

\item Infrastruktur: Softwerkskammer

\item Diskussionsgruppe: \\ \url{http://www.softwerkskammer.org/groups/craftsmanswap}

\item Firmensammlung mit Ansprechpartnern: \\ \url{http://www.softwerkskammer.org/wiki/craftsmanswap/an-craftsman-swaps-interessierte-firmen}

\item Musterverträge bereitstellen?

\end{itemize}

\end{frame}

%%%%%%%%%%%%%%%%%%%%%%%%%%%%%%%%%%%%%%%%%%%%%%%%%%
{
\usebackgroundtemplate{\includegraphics[width=\paperwidth,height=\paperheight]{background-slide.png}}
\begin{frame}{Vielen Dank!}

        Folien auf GitHub:
        \vspace{-0.8em}
        \begin{center}
                \url{https://github.com/NicoleRauch/CraftsmanSwaps}
        \end{center}

        \begin{block}{Matthias Mayer}
        \begin{description}[Twitterxx]
                \item[E-Mail]  \href{mailto:matthias.mayer@msg-gillardon.de}{\texttt{matthias.mayer@msg-gillardon.de}}
        \end{description}
        \end{block}
        \begin{block}{Nicole Rauch}
        \begin{description}[Twitterxx]
                \item[E-Mail]  \href{mailto:nicole.rauch@msg-gillardon.de}{\texttt{nicole.rauch@msg-gillardon.de}}
                \item[Twitter] \href{http://twitter.com/NicoleRauch}{\texttt{@NicoleRauch}}
        \end{description}
        \end{block}
\end{frame}
}
